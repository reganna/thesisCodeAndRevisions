\documentclass[12pt]{article}
\usepackage[utf8]{inputenc}
\usepackage{times}
\usepackage{mla13}
\usepackage{csquotes}
\sources{~/Desktop/thesis.bib}
\makeatletter
\newcommand\iraggedright{%
	\let\\\@centercr\@rightskip\@flushglue \rightskip\@rightskip
	\leftskip\z@skip}
\makeatother
\firstname{Randy}
\lastname{Gannaway}
\professor{Dr. Burningham}
\class{LAN}
\title{Thesis Proposal Draft}

\begin{document}
	\makeheader
	\iraggedright

\enquote{existe una rama de la crítica que atribuye el tremendo éxito de la novela a la sensibilidad hebrea de Isaacs} 
\enquote{la ambivalencia de Isaacs con respecto a cuestiones de raza y de herencia es no solo el tema subyacente en \textit{María} sino que también afecta su interpretación de las necesidades de los pueblos indígenas de Colombia}

 Colonizados como estan por
 el patriarcalismo espa¿ol y por los blancos que ahora gobiernan
 Colombia, estan destinados auna tragedia similar a la de Mariay Efrain,
 la tragedia de no poder incorporar una herencia pasada que los marca
 como racialmente inasimilables
\end{document}
