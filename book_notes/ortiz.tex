\documentclass[12pt]{article}
\usepackage[utf8]{inputenc}
\usepackage{times}
\usepackage{mla13}
\usepackage{csquotes}
\sources{~/Desktop/thesis.bib}
\makeatletter
\newcommand\iraggedright{%
	\let\\\@centercr\@rightskip\@flushglue \rightskip\@rightskip
	\leftskip\z@skip}
\makeatother
\firstname{Randy}
\lastname{Gannaway}
\professor{Lucía Ortiz}
\class{LAN}
\title{El negro y la creación romántico de una identidad nacional}

\begin{document}
	\makeheader
	\iraggedright

\enquote{es indudable que \textit{María} se representa una sociedad paternalista jerarquizada} \cite[362]{Ortiz2007}

\enquote{Isaacs describe una sociedad idílica y arquetípica, una sociedad feudal en extinción e inoperante en el momento del autor} \cite[362]{Ortiz2007}.

Ortiz equates the father to the societal role of law, morals, and religion. 

\enquote{la sociedad incluye al otro pero en una posición subordinada} \cite[363]{Ortiz2007}

\enquote{se apuntan implicita y explicitamente las diferencias que definen a los personajes de cada sector} \cite[364]{Ortiz2007}
\enquote{la mujer negro como objeto sexual}

\enquote{se enorgullecerían de sus raices hispánicas, de rendir culto a la gramática y de hablar el mejor español, ideales que formarían las bases para un proyecto nacional futuro} \cite[365]{Ortiz2007}

\enquote{En \textit{María} renace el Valle del Cauca como \textit{el paraiso} y la bondad de todas las criaturas} \cite[366]{Ortiz2007}

\enquote{Estos elementos otorgan el color local a la descripción pero también forman parte de los instrumentos lingüísticos manipulados por el autor romántico para embellecer el mundo creado en la novela} \cite[367]{Ortiz2007}.

Some critics see Isaacs representation as an effort to rescue an historical reality by using \enquote{el realismo supuestamente representado} \cite[367]{Ortiz2007}. The seeming realness of many characters, besides Efraín and María, lead these critics to take a realist approach to Isaacs representation and disregard much of what Ortiz would categorize as \enquote{figuras románticas idealizadas} (367)\nocite{Ortiz2007}. Isaacs frequent addition of details inspired by his life contributes to the semi-realism. However, the connection with reality does not compensate for the drastically glorified social harmony of Efraíns world.

%This is from a different Ortiz, but where???
"La utilidad de ambas obras para los proyectos de las élites políticas y literarias de Colombia contribuye también a explicar su suerte" \cite[141]{ortiz2002}."

"Críticos como Raymond Williams y Gustavo Mejía han señalado que María contribuía a apoyar un proyecto conservador que añoraba una arcadia católica y conservadora. Díaz, por el contrario, si bien criticaba la utopia liberal, no se afiliaba tampoco a los proyectos de su propio partido, el conservador, abriendo un amplio espacio para la expresión de los conflictos de clase y señalando la desigualdad social y las injusticias reinantesen la sociedad colombiana de su época" \cite[141]{ortiz2002}."

\end{document}
