\enquote{Una de las causas de esa conflictividad puede probablemente atribuirse a la accidentada geografía del país, que hizo difícil la comunicación entre las diferentes regiones, sus pueblos y ciudades permanecieron pobremente comunicados debido a los obstáculos topográficos que los separaban, y etso alentó el fortalecimiento de las lealtades locales y continuas rivalidades regionalistas} \cite[332]{Fox2011}.

\enquote{Se combinaron así los obstáculos en las comunicaciones y las rivaldades entre pueblos y municipalidades para dificultar el desarrollo de una nación unificada} \cite[333]{Fox2011}.

\enquote{En la costa del Caribe, en cambio, buena parte de la población indígena fue reemplazada por esclavos africanos; parte de ellos permanecieron en el área de la costa, otros terminaron trabajando en las minas y haciendas del interior, mayormente en el valle del Cauca} \cite[333]{Fox2011}.

%Fox comments on María
\enquote{la América de Chateaubriand es el marco exótico indispensable en la ficción romántica, mientras que la América de Isaacs es la tierra colombiana y la hacienda donde creció, y el elemento exótico es el folclore africano}. While the exoticism of the Nay and Sinar interlude is undoubted, nearly every recent critic has found it to be where Isaacs' explores some figurative treatment of Colombian political and social structures. 

