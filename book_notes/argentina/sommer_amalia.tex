Sommer introduces \textit{Amalia} as Mármols \enquote{rambling} serialized novel published in Montevideo's \textit{La Semana} in 1851 \autocite[83]{Sommer1991}.
However, as Civantos catalogs in her article, the publishing history of the full novel was as lengthy and disjointed as the final product \autocite[74]{Civantos2000}. 

\enquote{Both Mármol's conciliatory neo-Unitarianism and Hernández's reformed Federalism aimed to consolidate a nation rather than to defend provincial autonomy} \autocite[111-2]{Sommer1991}.

\enquote{Mármol's admirers, an elite class of literate Argentines returning from exile to take control at home, [...] could choose between hegemonic bonding and indulgent paternalism in his inconsistent formulations} \autocite[113]{Sommer1991}
Sommer notes that Mármol was an overnight success. 
Hernández, while popular, was not considered the founding Argentine novel until the earlier twentieth-century (113).

time and novel 106

race 96

desire allegory 


 
In her discussion, Sommer notes that the political efforts of Juan Bautista Alberdi championed an openness to religious and racial intermixing for the purposes of producing \enquote{legitimate, homegrown inheritors of local power and foreign capital} \autocite[103]{Sommer1991}.
While Echeverría's \textit{El matadero} left a longer lasting impression on the literature of Argentina, Sommer points out that it was his poem \textit{La cautiva} that \enquote{blasted open a new American literary terrain} \autocite[104]{Sommer1991}.
She claims that by \enquote{inscrib(ing) popular regionalisms}, one of several stylistic choices that were managing to \enquote{melt traditional literary barriers} \autocite[104]{Sommer1991}.
It is interesting that Sarmiento's founding work shares these characteristics with Hernández's supposedly ground breaking work.
As Chapter 2 pointed out, the focus on including new registers of language into popular fiction was a widespread phenomenon, apparent in countless nineteenth-century authors but more skillfully employed by some.
The distinction is articulated in Gala Blasco's summary of \textit{Martín Fierro} which claims that Hernández \enquote{no imita el lenguaje gaucho, como los escritores gauchescos: lo usa como idioma propioy con su propia sensibilidad concurre a enriquecerlo y formarlo} \autocite[400]{Blasco1991}.
In the Spanish speaking America's, vocabulary, which Sommer says \enquote{exceeds standard Spanish}, was the principal means for readers to recognize a familiar, yet unique, social identity within literatures that shared a common colonial language but not a common culture.
In fact she says of Argentina's generation of 1837, \enquote{the Romantics pointed out that their language was Argentine not Spanish} \autocite[105]{Sommer1991}.

Blasco et al. discuss the many linguistic changes that occurred in distinct geographical locations in the Americas.
One in particular is the solidification of the use of \textit{vos} throughout Argentina, a phenomenon that Blasco et al. assert may have a connection with the Rosas dictatorship (45).
They note that \textit{voseo} is found throughout Rosas' speeches and that \enquote{el auge y triunfo de rozismo coincide con la reinstauración del\textit{vos} entre quienes usaban el \textit{tú}} as well as pointing out that during this period other forms of address were also changing, something reflected in \textit{Amalia}(45).
In the novel, Rosas' sister-in-law María Josefa says \enquote{No me diga Usía [...] Ahora somos todos iguales [...] porque todos somos federales} \autocite[Mármol in][45]{Blasco1991}.

The role of literary language in national imagining is particularly salient in Argentina.
The complex dichotomy between an overtly European identity and the provincial gaucho image was exploited by Argentina's writers.
The success of Hernández's epic poem can be attributed to the readerly connection established through his use of \enquote{authentic} language \autocite[110]{Sommer1991}.
The balance of regional and standard vocabulary, I argue, was an important factor in determining a novel's success. 
The inclusion of regional vernacular did not ensure a successful readership, however, Blasco et al. point out \enquote{en ocasiones estos usos regionales limitaban la difusión de las obras, pero ello no arredraba a los románticos, quienes veían en este vocabulario parte de su alma genuinamente americana} \autocite[47]{Blasco1991}.
It is important to point out that this manipulation of the expectation of standard Spanish in literary registers was not an historical coincidence.
In Argentina the prominent role of language in political debates was reflected in the Romantic writers who \enquote{se ocuparon teóricamente de su forma de expresión como refejo de su independencia respecto de la metrópoli} \autocite[51]{Blasco1991}.
The inclusion of \textit{gaucho} vernacular was, more accurately, \enquote{una invención literaria creada por los primeros autores a partir de ciertas características reales} \autocite[48]{Blasco1991}.
It is this nostalgic attachment to regional vernacular that drives the most successful of founding fictions, and the reason that looking at literature under such broad a rubric as Latin American, can be misleading.