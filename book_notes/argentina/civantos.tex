\enquote{Many of the novels in the 19th century Latin American literature which most closely deal with the constitution of nation-states and national cultures were written in exile and/or thematize exile} \autocite[55]{Civantos2000}

% While Isaacs never was exiled per se his story definitely plays on the nostalgia of home, something he eventually accepts he will never recover.

emphasis on \enquote{la patria}

Discussion of the circumstances surrounding exile
specifically the meeting between Daniel and ...

\enquote{Having been imprisoned and pushed into exile by Rosas opened many doors for Mármol among the literary elite and essentially created a niche for him} \autocite[60]{Civantos2000}

The generation of Romantics (1837) that Mármol was a part of \enquote{they considered/imagined their country \enquote{desde un lugar incómodo por su excentricidad (ideológica y geográfica)}} \autocite[60]{Civantos2000}

Quote regarding Anderson's view on 60

\enquote{In \textit{Amalia} the nation is crossed by the conflicting desires to be European and gaucho --to be civilized as well as somewhat barbaric-- and in particular by the dominance of the \textit{patria chica}, or small homeland, of Buenos Aires} \autocite[60]{Civantos2000}

This patria chica is precisely the regionalism espoused by Isaacs

\enquote{Rather there is a consciousness, at least on the part of Daniel, that Argentina, such as it was, in its state of internal conflict and unfixed borders, arose from political circumstances} \autocite{Civantos2000}

\enquote{There is no attempt to present the nation as an \textit{a priori} organic unit, but rather only the desire to know if the formation of Argentina was only a product of temporary practical necessity or if it should be \enquote{a lasting effect} --a lasting political arrangement} \autocite[61]{Civantos2000}

Civantos points out Daniel's comment \enquote{la asociación en todo y siempre para ser fuertes, para ser poderosos, para ser europeos en América} \autocite[61]{Civantos2000}

searching europeos americanos in Davies yields no hits, google shows an early 19th c spike 

\enquote{at the time when Mármol wrote, Argentine writers were still looking to Europe for cultural orientation rather than exalting the \textit{gaucho} as the authentic \textit{criollo}} \autocite[62]{Civantos2000}

Chiaramontes examines the way words like \enquote{Argentina}, \enquote{argentino} and \enquote{porteño} were understood during the nineteenth-century \autocite[64]{Civantos2000}

\enquote{during the first decades of the 19th century, both before and after independence, the word \enquote{argentino} was equivalent to \enquote{porteño}} 
\enquote{after independence the term began to be used with a broader meaning only among the inhabitants of Buenos Aires} \autocite[64]{Civantos2000}

A conversation between Amalia and another unitarian woman reflects in the novel the broader tension between \enquote{patria chica} and \enquote{patria grande} \autocite[65]{Civantos2000}

Civantos points out the frequent reference to Amalia as \enquote{la tucumana} which highlights her provincial origin (65).

“Amalia aspiro hasta en lo mas delicado de su alma
todo el perfume poetico que se esparce en ei aire de su tierra natal" (Marmot
75-6) from civantos page 66
compare to the quotes from ISaacs about the qualities of caucana women

\enquote{in \textit{Amalia}, Marmol does make some attempt to discredit hard-line unitarianism and its concomitant privileging of the patria chica of Buenos Aires. However. the novel's picture ofthe nation never becomes particularly clear because. as we shall see in the discussion of Amalia below. the relationship between the provinces and Buenos Aires is ultimately tied to the issue of exile.} 

Although the co-existence of various potrias -:'ht'tvts could suggest a
more Àuid. plural formulation of the nation. it would only be so if the
emotional tics to the nation were also allowed to be free form 

Marmol’s text presents as accepted truth that it is natural that one’ s strongest
emotions be those for the patria and the aniada and creates a certain
parallelism between the two types of affection.
