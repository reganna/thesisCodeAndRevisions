\documentclass[12pt]{article}
\usepackage[utf8]{inputenc}
\usepackage{times}
\usepackage{mla13}
\usepackage{csquotes}
\sources{~/Desktop/thesis.bib}
\makeatletter
\newcommand\iraggedright{%
	\let\\\@centercr\@rightskip\@flushglue \rightskip\@rightskip
	\leftskip\z@skip}
\makeatother
\firstname{Randy}
\lastname{Gannaway}
\professor{Ericka Beckman}
\class{LAN}
\title{Fables of Globalization and Capital Fictions}

\begin{document}
	\makeheader
	\iraggedright

Capital fictions
"In the form lived today, under the hypercommoditized logic of neoliberalism, these fictions allow us to believe that the end of human existence is the market, not that the market exists to serve human needs" \cite[ix]{Beckman2012}.

My source texts also 
include economic essays predicting, in the 1870s, that Bolivia, Gua
-
temala, Colombia, and any number of nations were on the cusp of 
becoming rich beyond their wildest imaginations, a mode of dis
-
course I call “export reverie.” My corpus also includes little-studied 
objects of Latin American commercial culture such as banknotes 
and advertisements, alongside high artistic discourses like poetry \cite[x]{Beckman2012}


FABLES OF GLOBALIZATION
The chief aim of this essay is to move beyond the nation-state as a closed unit, to understand
how nineteenth-century Latin American writers allegorized a field of power
relations both internal and external to the nation: the global market \cite[99]{Beckman2013}

Hence if it is true that the nineteenth-century nation-state owes its very existence to the global market, Latin America’s long history of debt and financial instability would seem to show how the global market has also rendered the nation impossible (or at least, as has been the case at several moments in history, insolvent) \cite[100]{Beckman2013}

"While differing widely in form and style, these texts are united by two essential factors: first, by the centrality of transnational monetary relations to their plots and resolutions; and, second, by a common insistence on categories of race and sex as tools for understanding and making sense of the bewildering and destabilizing logic of market expansion. In all of the texts studied, that is, the inherent instability of race and sex provide a vocabulary for understanding the inherent instability of economic value in a transnational frame" \cite[101]{Beckman2013}

Beckman's reliance on texts that "remain largely forgotten today" reinforces the need for a broader analysis facilitated by DH(need a term) 

"The contrast (between characters in Un viaje al país de oro") sets up two kinds of personhood with respect to money, divided along lines of race: a kind of person who accumulates for the sake of accumulation, and one who accumulates reluctantly, only as a means to strengthen familial bonds" \cite[104]{Beckman2013}.

Is there a way to connect Gorritis perspective of international markets to the business/school relationships in Maria?
Specifically that Efraín must leave Colombia for the knowledge required to maintain the family estate.

"According to the logic of the text, excess is inherent in money, if we accept the resolution to this story, this excess can be neautralized by removing money from the erotically and racially charged realm of transnational exchange, and reinvesting it in the maternal "garden" of the nation state" \cite[105]{Beckman2013}.

"For the recruiting of the sexual charge of money through the channels of filial love (for a dead mother, no less) , means that the sexual process of reproduction through which monetary accumulation is allegorized in the novella has all but stopped" \cite[105]{Beckman2013}.
%compare this with the passages of Efraín's fathers illness and the talk of business
%possibly with the economic ruin in Sab
%it seems (so far) like domestic economic desperation is as much a theme as either international economy or erotic patriotism 
%is it possible that romance is an answer to economic instability and international economy offers opportunity
%another thought is that the romance is a reflection of historical literary movements and reading "around" the romance is necessary to find the national allegory

In analyzing the next novel La bolsa, Beckamns points out that the vital economic forces are endangered by "foreign parasites," "most importantly European jews"
%how can this be seen in the references to judaism in Maria?
There is also a principal Jewish character in the first novel

"What is different, of course, is that unlike the Spaniard, who ruled through brute force and territorial domination, Jewish domination is portrayed as taking place through invisible, but no less coercive, monetary relationships" \cite[107]{Beckman2013}.



