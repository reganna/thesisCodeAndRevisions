"equivalence across languages is not easy to achieve [...] this is especially the case in places where languages have particular social significance" \cite[177]{Vessey2014}.

Vessey studies Canada and notes that the language there serves important and divergent functions for different social groups \cite[177]{Vessey2014}.

Explores borrowed words and mock language, when words that have near equivalents are still taken from other languages (core borrowing) different from cultural borrowing which involves new lexical forms for objects of concepts new to the culture

"core borrowing is problematic for understandings of equivalence across languages because it is unclear why words are borrowed if translation equivalents already exist" \cite[177]{Vessey2014}.

%in the case of latin america the "borrowing" is adding regional words to the old world lexicon in order to assert that there is a difference between cultures, to make difference intentionally, those books that did this best created a deeper identification with regional readers

"core borrowings can only be understood insofar as both the referential meaning of a word and the connotations about the origins of that word are understood" \cite[177]{Vessey2014}.



indexing stereotypes about the other 178

"group boundaries that were previously based on are nowadays discussed in terms of "ethnicity" and "culture" \cite[178]{Vessey2014}.

%in s19 this could be seen as the need to make new boundaries from what was previously "new Spain", the added connundrum of a common language made separation along liguistic lines impossible, hence the need to highlight regional dialect in popular literature

"exploiting differences and contrasting it with a presumed homogeneous norm" \cite[178]{Vessey2014}.

with mock language the purpose may be to derrogate speakers of other languages, ostracize, in LA s19 it may or may not have this effect, ?is it ostracizing indigenous people so that new LAicans are not defined as them? ?is it seperating LA spanish from Spain so as not to be defined by them?

%was Isaacs "mocking" at the time of publication but later viewed as exalting during s20?

%I think that it displays both, for much of the "colombian" language is portrayed as acceptable but the critique by Efraín makes obvious a disdain for lessor vocabulary

"borrowed words may serve as covert means of signalling group boundaries" \cite[182]{Vessey2014}
%two directions in LA indigenous < identity > spanish
%class divide not ethinicity? education? <- this contributes to natl status of text

this is more applicable to the non-zero words produced by NHM 