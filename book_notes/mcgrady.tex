\documentclass[12pt]{article}
\usepackage[utf8]{inputenc}
\usepackage{times}
\usepackage{mla13}
\usepackage{csquotes}
\usepackage[draft]{todonotes}
\sources{~/Desktop/thesis.bib}
\makeatletter
\newcommand\iraggedright{%
	\let\\\@centercr\@rightskip\@flushglue \rightskip\@rightskip
	\leftskip\z@skip}
\makeatother
\firstname{Randy}
\lastname{Gannaway}
\professor{Donald McGrady}
\class{LAN}
\title{Introducción María}

\begin{document}
	\makeheader
	\iraggedright

In 1967, one hundred years after \textit{María} was initially published, the novel had been reprinted in around 150 editions \cite[13]{McGrady2012}.

Conncetion with other literature, Pablo y Virginia

Realism
\enquote{La sociedad representada en \textit{María} nada tiene de ideal, puesto que admite la institución repelente de la esclavitud} \cite[24]{Mcgrady2012}
\makeworkscited
\listoftodos
\end{document}
