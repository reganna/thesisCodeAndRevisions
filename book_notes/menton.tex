
\documentclass[12pt]{article}
\usepackage[utf8]{inputenc}
\usepackage{times}
\usepackage{mla13}
\usepackage{csquotes}
\sources{~/Desktop/thesis.bib}
\makeatletter
\newcommand\iraggedright{%
	\let\\\@centercr\@rightskip\@flushglue \rightskip\@rightskip
	\leftskip\z@skip}
\makeatother
\firstname{Randy}
\lastname{Gannaway}
\professor{Dr. Burningham}
\class{LAN}
\title{Book notes}

\begin{document}
	\makeheader
	\iraggedright

Points out immediately that while wildly popular, all did not agree (footnote p15)

"Las bodas inocentes de Bruno y Remigia y de Braulio y Tránsito representan el sueño platónico de Efraín y María menazado por el próximo viaje de él a Europa" \cite[24]{Menton1978}
I see this more like Sommer and how I think Beckman would, as allegory for the local political struggles of the planter class in the struggle against uprisings along race and class lines

so far Menton treats the Jewish-ness of the novel only briefly 

"el episodio de Nay y Sinar ha sido algo mal entendido pos los críticos quienes lo han considerado como muestra del gusto romántico por lo exótico" \cite[25]{Menton1978}

Nay and Sinar may also be, like Sommer mentions of other aspects, a displaced manner in which Isaacs treats the violence of mid-century Cauca

interesting observation about the comentary on the novia from bogota and her inability to adapt to the countryside by Carlos in comparison to what happens with Emigdio y Zoila who fears his father will object to the 'pueblerina' (28) 

"las descripciones [...] han contribuido a inmortalizar la geografía del valle del Cauca y de la ruta fluvial desde Buenaventura" \cite[29]{Menton1978}

Typical of the period is Isaacs use of natural description to reflect character mood. Menton observes that the antithesis between love and death is reflected in the representation of nature as "edénico" or "infernal" \cite[30]{Menton1978} The opposing locations being the edenic Cauca Valley and the trecherous Dagua river. Menton pens this dichotomy as a parallel between "paisaje natural" and "paisaje humano" which is manifested in the parallel between Efraín's love for María and his love for "la tierra" \cite[29]{Menton1978}.

McGrady also mentions the dark foreshadowing in several descriptions.

motifs of birds, flowers, water(rivers) p31-3

"en 1866 [...] apareció en Bogotá el \textit{Museo de cuadros de costumbres}, dos tomos de cuadros escritos por los miembros del círculo literario "El Mosaico." Isaacs que formaba parte del grupo, no pudo por lo tanto resistir la tentación de incorporar en su novela toda una serie de cuadros y de personajes costumbristas" \cite[34]{Menton1978}.

linguistically:
"Otro rasgo costumbrista empleado por Isaacs para individualizar a los personajes secundarios es su manera de hablar" \cite[35]{Menton1978}.

Menton discusses at length the various phonetic changes that Isaacs utilizes in order to distinguish the geography and class of his characters (35) \nocite{Menton1978}.copy this page

According to Menton, "\textit{María} es un documento hostórico y realista de la sociedad colombiana en las primeras decadas de la independencia (36) \nocite{Menton1978}.

Nota importante! The attitude of Don Ignacio towards schooling "los muchachos se echan a perder en los colegios de allá" and his opinion of emigdios plans -economy Aun peor don Jerónimo (p38)

Nota sobre la relación entre María y Nocturno de Silva

Menton closes mentioning the technique of \enquote{grupos binarios} such as \enquote{la casa, grande y antigua, rodeada de cocteros y mangos} \cite[48]{Menton1978}

\enquote{Esos valores históricos junto con su riqueza estructural y estilística dan a \textit{María} una vigencia actual que no merece ninguna otra novela romántica de lengua española}

\section{Chapter 3 Charrasquilla Frutos de mi tierra}
\enquote{\textit{textFrutos de mi tierra} sigue prácticamente desconocida. Publicada por primera vez en 1896, no había vuelto a publicarse hasta 1972}

\enquote{Además Carrasquilla sobresale por su gran dominio del idioma, la ingeniosidad con que moraliza y la destreza con que da un sentido nacional (pero menos que el de \textit{Manuela} y aún universal a su regionalismo)
	





\makeworkscited
\end{document}