\documentclass[12pt]{article}
\usepackage[utf8]{inputenc}
\usepackage{times}
\usepackage{mla13}
\usepackage{csquotes}
\sources{~/Desktop/thesis.bib}
\makeatletter
\newcommand\iraggedright{%
	\let\\\@centercr\@rightskip\@flushglue \rightskip\@rightskip
	\leftskip\z@skip}
\makeatother
\firstname{Randy}
\lastname{Gannaway}
\professor{Dr. Burningham}
\class{LAN}
\title{Thesis Proposal Draft}

\begin{document}
	\makeheader
	\iraggedright

Musselwhite claims that the time immediately prior to the publication of \textit{María} were \enquote{among the most turbulent and critical of Colombian nineteenth-century history} (41). 

on 41 he elaborates the pol/econ situation

Connect with Beckman

As a result of the Federalist Constitution of 1863 Colombia experienced an \enquote{increase in regional autonomy and factionalism} \cite[42]{Musselwhite2006}.

The novel mirrors Isaacs' displacement of Colombia's social disorder through the psychosomatic collapse suffered by Efraín's father upon hearing news of his financial ruin \cite[46]{Musselwhite2006}.

\enquote{That displacement is the fundamental trope of the novel has already been convincingly demonstrated by Doris Sommer: \enquote{In other words, \textit{María} used something of a narrative mechanism that Freud would call displacement as such}} \cite[46]{Musselwhite2006}

Musselwhite compares the Nay and Sinar interlude as allegory for the Colombian state. Specifically he compares Tomás Cipriano Mosquera to the African king Say Tuto Kuamima.





\end{document}
