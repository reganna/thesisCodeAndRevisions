\documentclass[12pt]{article}
\usepackage[utf8]{inputenc}
\usepackage{times}
\usepackage{mla13}
\usepackage{csquotes}
%\usepackage[style=apa]{biblatex}
\usepackage[backgroundcolor=lightgray]{todonotes}
\sources{~/Desktop/thesis.bib}
\makeatletter
\newcommand\iraggedright{%
	\let\\\@centercr\@rightskip\@flushglue \rightskip\@rightskip
	\leftskip\z@skip}
\makeatother
\firstname{Randy}
\lastname{Gannaway}
\professor{Dr. Burningham}
\class{LAN}
\title{Colombia Chapter}

\begin{document}
	\makeheader
	\iraggedright
\section{?Intro}
nationalism has to be understood by aligning it with the cultural systems that preceded it, the religious community and the dynastic realm(12)
\subsection{BLah}
"the religious community" 12-19
%these broad religious communities united through a sacred text that was written in one common language, not vernacular

"such classical communities linked by sacred languages had a character distinct from the imagined communities of modern nations"
"one crucial difference was the older communities' confidence in the unique sacredness of their languages, and thus their ideas about admission to membership" \cite[13]{Anderson2006}.

"Yet if the sacred languages were the media through which the great global communities of the past were imagined, the reality of such apparitions depended on an idea largely foreign to the contemporary western mind: the non-arbitrariness of the sign."
%This could be contrasted with work like Vessey's analysis of core borrowing that shows cross-over between languages of concepts that have signs in both, pointing to a view that the signs are not arbitrary and one fits a given context better than another

"the fall of Latin exemplified a larger process in which the sacred communities integrated by old sacred languages were gradually fragmented, pluralized, and territorialized \cite[19]{Anderson2006}. 

"the idea of a sociological organism moving calendrically through homogeneous, empty time is a precise analogue of the idea of the nation, which is also conceived as a solid community moving steadily down (or up) history" \cite[26]{Anderson2006}.

this in reference to the novel example:
"What then actually links A to D? Two complementary sonceptions: First, that they are embedded in societies"
"Second, that A and D are embedded in the minds of the omniscient readers" \cite[26]{Anderson2006}.

"the casual progression of this house from the interior time of the novel to the exterior time of the Manila reader's everyday life gives a hypnotic confirmation of the solidity of a single community" \cite[27]{Anderson2006}
%this connects nicely to commentary in Maria where Efraín breaks from the internal narrative
%leads me to wonder is this a characteristic of other romances nacionales?

in Periquillo he notes (31)
the horizon is clearly bounded, colonial mexico
the use of 'plurals' as 'sociological solidity'
%this could influence/be seen during classification or comparing freqs

he points out from a filipino text the importance of the man in the text reading the paper
to compare Efraín talks several times about readings and books, but never the newspaper

chapter 3
connecting capitalism with language and nation
Latin subsides and "the bulk of mankind is monoglot"
"The logic of capitalism thus meant that once the elite Latin market was saturated, the potentially huge markets represented by the monoglot masses would beckon" \cite[38]{Anderson2006}.

Three factors gave impetus to the "Revolutionary vernacularizing thrust of capitalism":
One - a change in the character of Latin itself
Two - the impact of the Reformation
Three - slow spread of vernaculars as instruments of administrative centralization

?Habsburgs and long lasting Latin <- compare with differences in Nation development/language

Last paragraph on 42 for summary 

Elements:
"For whatever superhuman feats capitalism was capable of, it found in death and languages two tenacious adversaries" \cite[43]{Anderson2006}.

notes that there is a "common element in nationalist ideologies which stresses the primordial fatality of particular languages and their association with particular territorial units" \cite[43]{Anderson2006}.

"had print capitalism sought to exploit each potential oral vernacular market, it would have remained a capitalism of petty proportions" \cite[43]{Anderson2006}.

"But these idiolects were capable of being assembled, within definite limits, into print-languages far fewer in number" \cite[43]{Anderson2006}.
%Then is a more enlightening search going to be that of texts with the higher combination of match with vernacular/colonial

print languages laid the bases for national consciousness in three distinct ways
One - they created unified fields of exchange and communication below Latin and above vernacular
%is this what happens in LA below Castellano and above Indigenous languages

Two - print capitalism gave a new fixity to language

Third - print capitalism created languages of power of a kind different from the older administrative vernaculars

Conclusion
"The convergence of capitalism and print technology on the fatal diversity of human language created the possibility of a new form of imagined community, which in its basic morphology set the stage for the modern nation" \cite[46]{Anderson2006}.

%did the novel introduce the intermediate category of print vernacular, which accounts for the fact that most of the regional terms in the larger corpora occur in fiction, not newspaper

It also seems like the newspaper played a key role in disseminating certain novels/authors as cultural artifacts 

\makeworkscited

\end{document}