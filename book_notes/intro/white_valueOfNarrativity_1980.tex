\documentclass[12pt]{article}
\usepackage[utf8]{inputenc}
\usepackage{times}
\usepackage{mla13}
\usepackage{csquotes}
\sources{~/Desktop/thesis.bib}
\makeatletter
\newcommand\iraggedright{%
	\let\\\@centercr\@rightskip\@flushglue \rightskip\@rightskip
	\leftskip\z@skip}
\makeatother
\firstname{Randy}
\lastname{Gannaway}
\professor{Hayden White}
\class{LAN}
\title{Value of Narrativity}

\begin{document}
	\makeheader
	\iraggedright

\enquote{it is the state which first presents subject-matter that is not only adapted to the prose of History, but involves the production of such history in the very progress of its own being}

\enquote{And  this 	suggests 	that 	narrativity, 	certainly 	in  factual 	storytelling 	and 	probably 	in  fictional 	storytelling 	as 	well, is 	intimately 	related  to, if not a function  of,  the 	impulse 	to moralize 	reality, 	that is, to 	identify 	it with the social 	system 	that is the source of 	any 	morality 	that we can 	imagine.} (18)

white's point about historical narrative being used to moralize events applies to Lit in that Lit is re-purposed thorughout history and reinterpretted by present culture in ways that past culture had intended or even required of it. María, as Sommer has elaborated, certainly provided an allegory of erotic love at one point in history to a certain audience. However, as Patiño's insight into Japanese migration shows us, the novel serves many purposes.

white later talks about authority and story, historical plot that \enquote{summon us to participation in a moral universe}. This is in fact why much of Isaacs' critique is present through its absence. At a time when Colombia is experiencing increased \enquote{regional autonomy and factionalism}, Isaacs' novel promotes the splendor of Cauca. Isaacs inclusion of many factual aspects in his novel make his intention of some historical interpretation quite evident. By not including certain historical milestones in his narrative, Isaacs reveals something fundamental about his intentions in writing it. 

According to White, in order to narratavize historical events, they must revolve around an authoritative moral center. He elaborates that \enquote{narrativity, certainly in factual storytelling and probably in fictional storytelling as well, is intimately realted to, if not a function of, the impulse to moralize reality, that is, to identify it with the social system that is the source of any morality that we can imagine} \cite[18]{White1980}. Isaacs lack of confidence in the direction of Colombia's authority and moral relativity is manifest in his elimination of current events from his narrative. 


aside
is this view of history fold into the evolution of nationalism by demonstrating how narrative, true or not, creates a moral center around which a group evolves

this aside is bad
It is interesting to note that the Japanese immigrants moved by María to travel to the valley of Cauca were not inspired by any supposed nationalist rhetoric but the by the picturesque countryside. While the imagery certainly inspired pride in \textit{vallecaucanos}, it inspired hope in distant immigrants. To Colombians, language played a uniting role, it made the novel feel purely Colombian. 
\end{document}
