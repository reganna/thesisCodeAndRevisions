\documentclass[12pt]{article}
\usepackage[utf8]{inputenc}
\usepackage{times}
\usepackage{mla13}
\usepackage{csquotes}
\sources{~/Desktop/thesis.bib}
\makeatletter
\newcommand\iraggedright{%
	\let\\\@centercr\@rightskip\@flushglue \rightskip\@rightskip
	\leftskip\z@skip}
\makeatother
\firstname{Randy}
\lastname{Gannaway}
\professor{Dr. Burningham}
\class{LAN}
\title{Book notes}

\begin{document}
	\makeheader
	\iraggedright
	

"analizar ciertos cambios en el uso del término nación en un lapso que va de mediados de los siglos XVIII a XIX \cite[27]{Chiaramonte2004}

"proyección anacrónica de nuestras preocupaciones actuales sobre el vocabulario político de otras épocas" \cite[28]{Chiaramonte2004}

"el nacionalismo ha tenido y tiene asi versiones compatibles con el supuesto de una relación armónica con otras naciones" \cite[28]{Chiaramonte2004} segue to Beckman

"la historia nacional [...] es una porción del patrimonio general que cada generación que desaparece lega a la que la reemplaza, ninguna debe transmitirla tal como la recibió sino que todas tienen el deber de agregar algo de certidumbre y claridad" \cite[28]{Chiaramonte2004}

"Las naciones no son algo natural, y los estados nacionales no han sido tampoco el evidente destino final de los grupos étnicos o culturales" \cite[][Gellner in]{Chiaramonte}

"se ha debatido con intensidad si las naciones tienen o no un orígen étnico" \cite[30]{Chiaramonte2004}
"uno de los motivos de más fuerte polémica en años recientes ha sido el criterio de rechazar la resis de los fundamentos étnicoos de las naciones, considerando que ellos no son una realidad sino una invención del nacionalismo"
see Hobsbawn, Gellner, Kedourie

points out "un equívoco concerniente a la datación del concepto político de nación"
"entre los mejores trabajos aparecidos recientemente subyace una confusión respecto de las relacions del concepto de nación con la revolución Francesa" \cite[31]{Chiaramonte2004}
cites Sieyés and Renán as the first of such definitions
continues that "suele ir unido al concepto de un nexo entre esa idea de nación y el ascenso de la burguesía" \cite[32]{Chiaramonte2004}

cites andersons approach on page 32
Anderson cites Renán page 6

"Es posible interpretar que la dominante preocupación por el nacionalismo en la historiografía europea ha llevado a superponerla historia del movimiento de expansión de los Estados nacionales a la historia de los conceptos sustanciales al nacionalismo, como el de nación"
cites Kohn, Anderson, and Hobsbawn as most recent scholarship in which this tendency is observed

España siglo XVII "se usaba el concepto de nación 'a la manera antigua' aplicándolo a gente de un mismo orígen étnico"

this supports the still racial tensions in nationalist fictions and fables

page 33 he notes the replacement of república as a synonym for nación y estado which shows an inverse relationship when plotted in ngram viewer

este sinónimo "despoja (strips) del sentido étnico del concepto de nación de su antiguo contenido étnico" \cite[33]{Chiaramonte2004}
this affirms that the political connotation of nation existed prior to the french revolution

"el siglo XVIII nos ofrece un uso doble del término nación: el antiguo, de contenido étnico, y el que podemos llamar político, presente en la tratadística del derecho natural moderno y difundida por su intermedio en el lenguaje de la época" \cite[37]{Chiaramonte2004}

interesting distinction between the concept of national and federal, federal not coming from individuals
he quotes ??? gobierno federal "los poderes actúan sobre los cuerpos políticos que intergran la Confederación, en su calidad política"
 en el nacional "sobre los ciudadanos"
 
Dice Chiaramonte que según Hobsbawn "el uso del lenguaje común constituyó un requisito para la adquisición de la nacionalidad, aunque en teoría no la definía" \cite[39]{Chiaramonte2004}.

"todavia en el siglo XIX se distinguía los grupos de esclavos africanos por "naciones" \cite[40]{Chiaramonte2004}

cite GORRITI and Feijóo definition of nation and hispanoamérica in congress \cite[41]{Chiaramonte2004}
 
"el rasgo más significativo, para nuestro objeto, que subyace en el análisis de Hastings desde un comienzo, es la postulación de la nación como una realidad intermedia entre grupo étnico y Estado nacional" \cite[45]{Chiaramonte2004}
"es el punto débil de este tipo de análisis" "que genera distinciones demasiados simples como la explicación del paso de la etnia a la nación" por "la aparición de una literatura vernácula, particularmente por la traducción de la Biblia a las lenguas romances" \cite[45]{Chiaramonte2004}

This looks alot like Anderson's slow spread of vernaculars as instruments of administrative centralization p40-1

NOPE Hastings is a priest maintaining that the BIBLE is the printed resource that unites this whereas Anderson specifically notes that the end of religious unification (linguistically) is a step towards other "imagined communities"

Notes that there existed a time that nation carried a regional connotation much like city, province, etc.

this might carry over into the regionalism of LA lit in s19 where Cauca is seen as "community" seperate from say "Antioqueño", something that would persist into regional conflicts of the mid s20 colombia

\makeworkscited
\end{document}