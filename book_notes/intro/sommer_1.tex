
	
"It is worth asking why the national novels of Latin America-- the ones that governments institutionalized in the schools and that are now indistinguishable from patriotic histories-- are all love stories. An easy answer, of course, is that nineteenth-century novels were all love stories in Latin America" \cite[30]{Sommer1991}.

	This needs addressed... Why can't it be a chronological phenomenon and these novels exhibits characteristics beyond romance that have positioned tham in the 		foundational canon.
	The linguistic/lexical contribution of these works shouldn't be overlooked

These novels were excluded from the first national literary histories. The "programmatic centrality of novels came a generation later [...] afer renewed internal oppositions pulled the image of an ideal nation away from the existing state" \cite[30]{Sommer1991}.

"each romance shares far more than its institutional status with others. The resemblances may be symptomatic of nationalism's general paradox; that is, cultural features that seem unique and worthy of patriotic (self)-celebration are often typical of other nations too and even patterned after foreign models" \cite[31]{Sommer1991}.
%This seems to be the point that our views part

	The question I would ask then is how much similarity or difference do they actually display when compared with:
		1. other writings at the time (story, poem, news, etc.)
		2. other novels of equal standing from other countries

Sommer states that one purpoese of her volume is to "account for the generic coherence that individual readings will necessarily miss" \cite[31]{Sommer1991}.

Nancy Armstrong is quoted in Sommer saying "the formation of the modern political state-- in England at least-- was accomplished largely through cultural hegemony, primarily through the domestic novel." Sommer continues "this is possibly true for Latin America as well, where, along with constitutions and civil codes, novels helped to legislate modern mores" (33).

Sommer compares Foucault and Anderson whose books she says overlap the question of desire and patriotism as "timeless and essential to the human condition" (33).
These ideas intersect at their origin which each author claims to be the end of the eighteenth century \cite[33]{Sommer1991}.

She notes that Foucault makes important insights but fails to acknowledge heterosexual exhibitionism, the novel, and the invention of modern states \cite[37]{Sommer1991}.

Anderson in Sommer "nationalism is not 'aligned' to abstract ideologies such as liberalism or marxism but is mystically inflected from the religious cultural systems 'out of which - as well as against which- it came into being" (37).

sexuality and racism (38)

"Unlike Foucaults dour tracing of sexuality to a priestfood of moralizers and pseudo-scientists, Anderson locates the production of nationalsm precisely in the space of our democratically shared imagination, the private spce of novels that links us serially and horizontally through a 'print community' \cite[39]{Sommer1991}.
	
	Another reason to suggest that mining newspapers could provide insight.


Anderson "doesn't discuss the passions constructed by reading novels, or their ideal gender models that were teaching future republicans to be passionate in a rational and seductively horizontal way" \cite[40]{Sommer1991}.

Sommer defines allegory as a "narrative structure in which one line is a trace of the other" but admits a more standard interpretation as "a narrative with two parallel levels of signification" (42).

Sommer notes that the foundational fictions are "philosophically modest, even sloppy" and also says that, with the excepcion of María, these novels "do not actively worry about any incommensurability between Truth and Justice" (45).

"foundational novels are precisely those fictions that try to pass for truth and to become the ground for political association" \cite[45]{Sommer1991}.

	can someting of this difference be seen through topic modeling? Would María produce a more polemic list of discourses?

"If we 'know' from reading Amalia that Rosas was an unscrupulous dictator, our knowledge is to a considerable degree a political articulation of the erotic frustration we share with Amalia and Eduardo. And we feel the intensity of their frustration because we know that their obstacle is the norrible dictator" \cite[47]{Sommer1991}.


"José Hernandez developed an already existing genre of politically conciliatory poems that, as Josefina Ludmer masterfully shows, constructed a national voice by appropriating the language of 'authentic' but notoriously shiftless Argentines for patriotic and economically rational projects" \cite[111]{Sommer1991}.

\section{Chapter 6}
"\textit{María's} canonical status is surprising, almost perverse" \cite[172]{Sommer1991}.

"\textit{María} neither projects futures nor finds any obstacle that it might hope to overcome" \cite[172]{Sommer1991}.
I disagree with this, depending on the rest of the article, I think it does make subtle statements about obstacles and the future
after all it was Efraíns leaving to become a doctor, see ciudad letrada, that kills Maria... this issupported I think by a discussion between Efrain and his dad about his education in Europe bringing wealth and stability to the family 

on the hebrew aspect of the novel (to which Beckman has a different response)
"I suspect that before 1867 the flamboyantly Hebrew name may have been as foreign as the converso father that chose it for his son" \cite[173]{Sommer1991}.


It appears in Davies corpora more prominently in the 1600's, disappears in the 1700's and grow steadily thereafter 
interestingly in the 1900's it appears predominantly in the newspapers
re wikipedia Jew came to Colombia from Jamaica at the end of s18 and Judaism was made legal after independence, it also points out that the families were concentrated in the Cauca valley 

"the problem is being Jewish, a double bind that becomes Isaac's vehicle for representing a dead end for the planter class" \cite[173]{Sommer1991}

"these novels demand a possible solution o failed romance (read also national progress and productivity)" \cite[174]{Sommer1991}.

"it has no apparent political or social causality, no racial hatred, no regional conflicts" \cite[174]{Sommer1991}.


Again, disagree, I think Isaacs subtle treatment of Colombian political and social dynamics is less apparent because it was written for a Colombian audience, even more specifically, an audience familiar with Cauca

180 - 182 Implications of the Cons/Lib divide mid-century and the fall of the 'plantocracy'
"on a second reading at least, is about the end of an entire social system"

185-6 Jewishness in the novel and Colombia

"at one point Efraín indulgently, or disdainfully, explains his father's superstition as a vestige of his Jewishness"

"Isaacs is most ambivalnent in his description of María herself" \cite[192]{Sommer1991}

"Apparently, the narrator (and Isaacs), is caught between the poles of ethnic identification, unsure whether "Jew" is a religious affiliation that on can convert out of ot a race that is biologically and indelibly fixed" \cite[194]{Sommer1991}.

Great connection to Chiaramonte and the discussion of the ethnic connotations of "nation"

194 The discussion with Transito about white people riding horses

197 -
Sexuality, Marías illness as histeria
Efraín must choose between Domestic cure (sex) and conservative cure (restraint) he opts for filial loyalty
Sommers analogizes this to Colombias planter versus black conflict
Efraín's "caution or cowardice has everything to do with Colombias's national frustrations. The question for Isaacs class was whether to satisfy the blacks' and the liberal whites' desire for change, or to control those desires and forestall a racially mixed, possibly monstrous progeny" \cite[198]{Sommer1991}.

through metonymy Isaacs displaces "race riots for María's epilepsy" \cite[200]{Sommer1991}

the tale of Nay and Sinar takes it to metaphor
"as if Isaacs were purposefully pointing our symptomatic reading back to the obsessive experience he cannot mention (race riots)" \cite[200]{Sommer1991}

"Isaacs was not displacing one fearsome race for another more promising on in the interest of constructing a national myth. On the contrary; he seems to be saying that no myth of amalgamation is possible, because the patriarchal world he yearns for will not have it"
"he displaced the inassimilable black masses and the anachronistic planters onto his innocent but flawed Jewish heroine" \cite[202]{Sommer1991}