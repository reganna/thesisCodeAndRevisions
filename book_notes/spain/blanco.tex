\documentclass[12pt]{article}
\usepackage[utf8]{inputenc}
\usepackage{times}
\usepackage{mla13}
\usepackage{csquotes}
\usepackage[draft]{todonotes}
\sources{~/Desktop/thesis.bib}
\makeatletter
\newcommand\iraggedright{%
	\let\\\@centercr\@rightskip\@flushglue \rightskip\@rightskip
	\leftskip\z@skip}
\makeatother
\firstname{Randy}
\lastname{Gannaway}
\professor{}
\class{LAN}
\title{}

\begin{document}
	\makeheader
	\iraggedright
	
A re-examination of the literature of the 19th century may at first glance appear redundant given the quantity of analysis it has already received. 
However, leaving aside the application of new analytical techniques, Alda Blanco argues that there still exists a need for a renewed approach to the mid-nineteenth century in traditional scholarship \cite[423]{Blanco2000}.

\enquote{The polemic about the novel registers the preocupations and the anxieties of the \enquote{hombres de letras} who lived in a society and culture that was in the process of reconfiguring itself as a bourgeois nation-state} \cite[429]{Blanco2000}.

Nocedal furthermore insists ,that the novel be verisimilar, particularly in relationship to the two categories which he considers to be the most
problematic and dangerous in the novelistic text: the representation of gender and class relations \cite[430]{Blanco2000}.

Nocedal demands that novelistic verisimilitude be the representation of that which is “naturally” complementary between the sexes and among the social classes. 
Therefore he argues that the novel should not depict gender and class relationships as conflict or struggle \cite[430]{Blanco2000}.


\makeworkscited
\listoftodos
\end{document}
