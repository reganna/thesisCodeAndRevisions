\documentclass[12pt]{article}
\usepackage[utf8]{inputenc}
\usepackage{times}
\usepackage{mla13}
\usepackage{csquotes}
\usepackage[draft]{todonotes}
\sources{~/Desktop/thesis.bib}
\makeatletter
\newcommand\iraggedright{%
	\let\\\@centercr\@rightskip\@flushglue \rightskip\@rightskip
	\leftskip\z@skip}
\makeatother
\firstname{Randy}
\lastname{Gannaway}
\professor{}
\class{LAN}
\title{}

\begin{document}
	\makeheader
	\iraggedright
	
\enquote{if one turns to critical appraisals of the work for indications of the qualities that presumably have accorded \textit{Trafalgar} its privileged standing, one finds that the tacit assertions of superior aesthetic value that might be expected are curiously absent} \cite[183]{Iarocci2003}.

Iarocci points out that even the conception of the starting point for the \textit{episodios nacionales} has is origins in Galdós' awareness of the subconscious workings of language.
Galdós is quoted saying \enquote{cuando me preguntó en qué época pensaba iniciar la serie, brotó de mis labios como una obsesión del pensamiento, la palabra Trafalgar} \cite[184]{Iarocci2003}.
He elaborates that the emphasis on the word, by making it the subject of \textit{brotó}, puts the writer in the background, highlighting the importance of language in his project.

\enquote{The story of modern Spain continues to begin somewhere within the vicinity of the War of Independence (1808-1814)} \cite[185]{Iarocci2003}.

Iarocci points out that the use of first person in \textit{Trafalgar} is particularly effective at reinforcing the connection between the individual and the events of history (189).
Iarocci also points out the importance of not only viewing this as a \enquote{history} but also the \textit{bildungsroman} of Gabriel Araceli \cite[189]{Iarocci2003}.
Seen as a novel of formation, the love allegories of Latin America appear more closely relatable to Galdós' work.

\enquote{an analysis of the interconnections between the sentimental and historical plots} helps reveal \enquote{the preoccupation with the social status and meaning of masculinity} \cite[190]{Iarocci2003}.


\makeworkscited
\listoftodos
\end{document}
