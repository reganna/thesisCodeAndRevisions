Embree

Schutz said \enquote{man finds himself from the outset in surroundings already mapped out by Others} \cite[21]{Embree1998}

\enquote{Thus, his biographical situation in everyday life is always an historical one because constituted by the sociocultural processes which had led to the actual configuration of this environment} \cite[22]{Embree1998}.

It is important to take into consideration as well that an author's biography, style, or intention are limited in the extent to which they can influence a reader's interpretation.
Schutz stated about the social relationship between author and audience: \enquote{Any work of art, once accomplished, exists as a meaningful entity independent of the personal life of its creator} \cite[133]{Nasu1998}.
As mentioned previously this is precisely the relationship that allowed \textit{María} to become an inspiration to Japanese migrants while serving Caucan landowners as a lament for the past.
Nasu warns however that understanding the author is key to participating in the social relationship that the author intended \cite[134]{Nasu1998}.

This leads to two implications in the examination of the role of the novel on nation formation. 
First, the readers knowledge of the authors intention will always be imperfect.
This increases as the temporal distance between author and reader grows.
Second, any information the reader receives is filtered through historical interpretation. \todo{add white}
Often this involves an intentional re-interpretation of literature to better serve a contemporary social project.
Such is the case with vacillating attitudes towards indigenous language throughout the late 19th and early 20th century, another phenomenon that helped novels like \textit{María} to play different roles throughout history. 
