\documentclass[12pt]{article}
\usepackage[utf8]{inputenc}
\usepackage{times}
\usepackage{mla13}
\usepackage{csquotes}
\usepackage[draft]{todonotes}
\sources{~/Desktop/thesis.bib}
\makeatletter
\newcommand\iraggedright{%
	\let\\\@centercr\@rightskip\@flushglue \rightskip\@rightskip
	\leftskip\z@skip}
\makeatother
\firstname{Randy}
\lastname{Gannaway}
\professor{}
\class{LAN}
\title{}

\begin{document}
	\makeheader
	\iraggedright
	
	\enquote{Me representé a mi país como una inmensa tierra poblada de gentes, todos fraternalmente unidos} \cite[?]{Galdós, trafalgar}.
	
	\enquote{Nationlism is inseperable from modern historical thought. The nineteenth century envisioned historical development in terms of national entities} \cite[15]{Rodriguez1967}
	
	\enquote{Many Romantic Novels are consequently limited to indirect outpourings of national feeling} 16
	\enquote{In \textit{Trafalgar}, from the very onset of the work, the reader is made to witness the birth of national consciousness}
	This is where Galdós idealizes his own \enquote{realistic} version of history. During the nineteenth century just as many groups in Spain were striving to separate from the Spanish identity as were trying to bolster it.
	Galdós also emphasizes in his vision of Spain \enquote{a dynamic complex of future possibilities} \cite[17]{Rodriguez1967}
	
	
	\enquote{Araceli seeks the fulfillment of his impossible love; and the plot highlights the lovers' various encounters} 55
	
	


\makeworkscited
\listoftodos
\end{document}
