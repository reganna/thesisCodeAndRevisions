\documentclass[12pt]{article}
\usepackage[utf8]{inputenc}
\usepackage{times}
\usepackage{mla13}
\usepackage{csquotes}
\usepackage[draft]{todonotes}
\sources{~/Desktop/thesis.bib}
\makeatletter
\newcommand\iraggedright{%
	\let\\\@centercr\@rightskip\@flushglue \rightskip\@rightskip
	\leftskip\z@skip}
\makeatother
\firstname{Randy}
\lastname{Gannaway}
\professor{Diane F. Urey}
\class{LAN}
\title{La historia y la lengua en la primera serie de los \textit{Episodios nacionales} de Galdós}

\begin{document}
	\makeheader
	\iraggedright
	
\enquote{Galdós revitalizó y revolucionó al lector español, hasta entonces acostumbrado a folletines y malas traducciones} \cite[1525]{Urey...}

Dice en cuanta al Marcial \enquote{su lenguaje es <<mareante>> en su confusión constante de términos y definiciones} \cite[1526]{Urey}

\enquote{establece uno de los muchos paralelos importantes a Gabriel, que, como varios críticos han observado, es una figura picaresca, y que, a su edad, tampoco es todo un hombre} \cite[1526]{Urey}
This helps to highlight a tendency, as the 19th century wore on, for authors from the various Latin American republics to depend less on the repetition of old styles and tropes.
This emphasis in a stylistic separation from their colonial past comes to fruition with the contribution of \textit{modernismo} as the first style attributed with originating in an old Spanish colony.

Urey points out that Gabriel's role as narrator and character narrated put emphasis on the metafictional role of \textit{Trafalgar} \cite[1526]{Urey}.

Another parallel between the work of Isaacs and Galdós' \textit{Trafalgar} is the prominence of description.
Urey points out that Galdós intentionally puts emphasis \enquote{al acto de representar que al mundo representado} \cite[1527]{Urey}.

An observation unique to \textit{Trafalgar} is that individual characters serve as metaphors for particular time periods in Spain's history \cite[1527]{Urey}. 

Urey compares one passage of Gabriel's war games with Lazarillo del Segundo tratado
Sieber demonstrates that the connection of language, be it fictional or not, and identity has long been a component of literature \nocite{Sieber1978}(...)



\enquote{Para crear una identidad estable, de honor, que él controle, como sucede al final de la Serie, tiene que aprender a controlar su propia identidad verbal} \cite[1531]{Urey}
Urey goes so far as to draw a striking parallel, noting that \enquote{al final de la Serie Gabriel [...] ha aprendido a controlar el juego de lenguaje que constituye su texto} which she claims reflects Spain's ability to recapture it's own national identity \cite[1532]{Urey}.

Every version, be it novel or history, \enquote{no es otra cosa que una interpretación más, una parte de una totalidad que nunca se puede recapturar, si existiera originalmente} \cite[1533]{Urey}

\makeworkscited
\listoftodos
\end{document}
