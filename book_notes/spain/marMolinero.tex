\documentclass[12pt]{article}
\usepackage[utf8]{inputenc}
\usepackage{times}
\usepackage{mla13}
\usepackage{csquotes}
\usepackage[draft]{todonotes}
\sources{~/Desktop/thesis.bib}
\makeatletter
\newcommand\iraggedright{%
	\let\\\@centercr\@rightskip\@flushglue \rightskip\@rightskip
	\leftskip\z@skip}
\makeatother
\firstname{Randy}
\lastname{Gannaway}
\professor{mar-molinero}
\class{LAN}
\title{politics of language in spanish nationalism}

\begin{document}
	\makeheader
	\iraggedright
	
	According to Clare Mar-Molinero, beyond linguistic analysis, the identifying power of language is the only explanation for the effort that humans put into maintaining thousands of mutually incomprehensible modes of communication \cite[2]{marMolinero2005}.
	She claims that \enquote{the answer must lie in an innate need and desire to protect difference across groups and communities} \cite[2]{marMolinero2005}.
	Mar-Molinero goes on to clarify that \enquote{few if any marked-out states are naturally monocultural} later noting that the attempt to usurp this cultural diversity in the nineteenth century \enquote{laid the seeds for many explosive separatist movements} \cite[10]{marMolinero2005}.
	
	\enquote{In the vast majority of cases states are not congruent with nations} \cite[4]{marMolinero2005}.
	
	\enquote{The hostory of nation-building is however so often that of the triumph of the majority or the most powerful who have swept minority communities to one side in order to create a monocultural society for thier state} \cite[4]{marMolinero2005}
	
	Notes the dichotomy in categorizing different kinds of national, civic versus ethnic
	
	According to Fishman \enquote{vernacular literature has played a role in representing to national communities their \enquote{linguistic \textit{differentiation} and literary \textit{uniqueness}} \cite[Fishman in][9]{marMolinero2005}.
		



\makeworkscited
\listoftodos
\end{document}
