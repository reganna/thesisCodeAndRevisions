


"stories thereby provide one of the preeminent external storage systems for individual as well as collective memories [...] narrative also affords a primary "modeling device" or way of operating on, (re)configuring, and reasoning about those prior experiences." \cite[229]{Herman..?}

- 1 issue here Over time by cuing and reinforcing such inferences, stories can
just regular or recurrent but furthermore natural and normal. The connecf
tions will then seem as though they are built into the very structure of
things, instead of being the product of human assumptions, mores, and
practices.”
ln this sense, narrative not only constitutes an instrument of mind but
also affords an important object for ethnographic (iii particular, COg1"il’CiVEff."""""""""""
degree to which ways of thinking vary across time and space. A number
that no narrative can exhaustively specify causal connections between the
situations, actions, and events that it recounts, what pcitteriis of CL?L15ClZI,2_fii-I;Q§.'E3§_é§Iif5-_li§§§§;§§; ?
widei'speci¿cati0ii surface in a given story or kind of story? Further,
the patterns overlap with those with which the analyst is natively
these pairs is an effect caused by the first inember. But there is a de6PeIifi5fiijiii-iiii':%§:§:§
work to make a wide variety of (contingent) causal connections seem no
of research questions can be posed in this connection. Poi example, given:
int; that such Patterns support the co11'1p1'ehension of, ¿nd
within, and are in turn propagated by narrative texts, to what extent
-
heuristic totrigger the inference that the second member of each Of

anthropological) study-—that is, for forms of inquiry that explore ._ (238)

Also he talks about how the interpreter of narrative considers how the patterns it presents overlaps with those of their own culture (239)

"How do narratives at once derive from and perpetuate the 'common sense' by which members of a culture regularly judge one anothers attitudes, dispositions and pracitces" (239)? 
	I could see the commentary on history filling in this gap. The paradox derivation and perpetuation is possible because of the hsitorical reconfiguration acknowledged on page 229.


 Brunei (1991) discusses related ideas under the rubric of “canonicity
and breach,” noting that “to be worth telling, a tale must be about how
an implicit canonical script has been breached, violated, or deviated from
in a manner to do violence to the ‘legitimacy’ of the canonical script” \cite[239]{Herman}
	Could the breach in legitimacy for the LA found...fict be the departure from traditional linguistic patterns?

See Schutz and typicfication same page
Schutz thus suggests that the process of typification cuts across a variety of
sense-making activities—-from the organization of objects into classes and
members-of-classes, to the learning of the lexical and syntactic patterns of
a language