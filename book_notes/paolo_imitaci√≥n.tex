\documentclass[12pt]{article}
\usepackage[utf8]{inputenc}
\usepackage{times}
\usepackage{mla13}
\usepackage{csquotes}
\sources{~/Desktop/thesis.bib}
\makeatletter
\newcommand\iraggedright{%
	\let\\\@centercr\@rightskip\@flushglue \rightskip\@rightskip
	\leftskip\z@skip}
\makeatother
\firstname{Randy}
\lastname{Gannaway}
\professor{Osvaldo Di Paolo}
\class{LAN}
\title{Imitación, asimilación, importación y posesión: una lectura cursi de \textit{María} de Jorge Isaacs}

\begin{document}
	\makeheader
	\iraggedright

\enquote{it illustrates the desire of the Colombian bourgeois for wealth, elegance and the accumulation of objects} \cite[6]{Paolo}

\enquote{En	Colombia, el movimiento romántico se distingue por una tendencia conservadora y un énfasis en el intimismo, sobre todo la novela, género en el que se produce una separación entre las actividades políticas y las obras literarias.} \cite[6]{Paolo}

The exaggerated sentimentalism of \textit{María} has lead Osvaldo Di Paolo to explore \enquote{una lectura cursi} of Isaacs work. He notes the origin of the word \textit{cursi} as being employed to \enquote{desairar a los burgueses venidos a menos que tratan de simular lo que no son} \cite[7]{Paolo}. \blockquote{En el caso de España y de América Latina, la palabra cursi se asocia con un individuo que intenta ser ostentoso y que pretende vincularse con cierta clase, pero que no lo consigue hacer de forma convincente. Esta persona y sus cosas frívolas forman lo cursi.} Di Paolo points out that romanticism demonstrated a particular propensity for \textit{cursilería} because it attracted \enquote{una clase media que pretende imitar al burgués rico confundiendo lo material y lo sentimental} (7).

In talking about the phenomenon in LA Di Paolo references Angel Rama

\enquote{Rama ve al modernismo dentro del fenómeno de la modernización y lo data a partir de 1870 debido al aumento del imperialismo europeo y norteamericano} \cite[8]{Paolo} 

\enquote{Si se tiene en cuenta que   Issacs escribe María en 1867, tan sólo 3 años antes de la fecha en que Rama ubica el comienzo del modernismo, es posible detectar en  la  novela  una  porosidad  donde  se  comprueben  elementos modernistas definidos por el comportamiento cursi del burgués,} \cite[8]{Paolo}

Di Paolo discusses the new class mixture caused by recent independence in Latin America. Of particular interest, he highlights that \enquote{En la entonces Santa Fe de Bogotá, este naciente patriciado se esfuerza por distinguirse de la gente de \enquote{ruana}} \cite[8]{Paolo}.

This use should connect well with Vessey and core-borrowings in the sense that \enquote{ruana} is being taken and adapted with a negative connotation.

Cites \enquote{things} listed in the novel page 9

He notes that Emigdio \enquote{es un personaje cursi por su deseo de imitar el comportamiento de los bogotanos, en su afán de integrarse al sector de la sociedad que concentra el flujo de capital} \cite[9]{Paolo}. This distinction between the capital and country repeats itself throughout the novel.

\enquote{La obligacón de satisfacer los mandatos de su padre trae consigo el aparentar y desear la educación y los comportamientos sociales europeos; entre ellos, el modo de vestir refuerza la imitación y la asimilación implícitas en el comportamiento cursi de un individuo y denotan el deseo de la familia de Efraín de entrar en la infraestructura económica centralista} \cite[9]{Paolo}.

According to Di Paolo the women, María most of all, attempt to imitate a european fashion, a point made apparent in Efraín's extended descriptions of her clothing \cite[9]{Paolo}.

This fixation on material objects and romance Di Paolo insists provides an escape in uncertain times. 
\end{document}
