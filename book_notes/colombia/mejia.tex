\documentclass[12pt]{article}
\usepackage[utf8]{inputenc}
\usepackage{times}
\usepackage{mla13}
\usepackage{csquotes}
\usepackage[draft]{todonotes}
\sources{~/Desktop/thesis.bib}
\makeatletter
\newcommand\iraggedright{%
	\let\\\@centercr\@rightskip\@flushglue \rightskip\@rightskip
	\leftskip\z@skip}
\makeatother
\firstname{Randy}
\lastname{Gannaway}
\professor{}
\class{LAN}
\title{}

\begin{document}
	\makeheader
	\iraggedright
	
 Dicha hacienda de La Manuelíta, al igual que otra adyacente y
 que se conocía como La Rita, perteneció a la familia Isaacs-Ferrer hasta
 el año de 1864, cuando ambas fueron compradas por el norteamericano
 Santiago Eder,
 
 Nos proponemos mostrar de qué manera en la novela de Isaacs -en
 parte a través del papel que personalmente le tocó jugar al autor en
 ese conflicto histórico- podemos encontrar las repercusiones de los
 acontecimientos que, aproximadamente en 1850, iniciaron la transforma-
 ción del modo de producción heredado de la colonia, caracterizado por
 "las prohibiciones mercantilistas para evitar el desarrollo de manufacturas
 nativas”.2
 
 La historia del medio siglo XIX colombiano está determinada por el
 espacio sociopolítico creado por el advenimiento al poder del Partido Liberal,

\makeworkscited
\listoftodos
\end{document}
