\documentclass[12pt]{article}
\usepackage[utf8]{inputenc}
\usepackage{times}
\usepackage{mla13}
\usepackage{csquotes}
\usepackage[backgroundcolor=lightgray]{todonotes}
\sources{~/Desktop/thesis.bib}
\makeatletter
\newcommand\iraggedright{%
	\let\\\@centercr\@rightskip\@flushglue \rightskip\@rightskip
	\leftskip\z@skip}
\makeatother
\firstname{Randy}
\lastname{Gannaway}
\professor{Dr. Burningham}
\class{LAN}
\title{Colombia Chapter}

\begin{document}
	\makeheader
	\iraggedright
\enquote{La riqueza lingüistica de Colombia era aún mayor hace 500 años, pero una política de opresión lingüística colonial y republicana, deliberadamente encaminada a erradicar su existencia, trajo como consecuencia la extinción de un gran número de lenguas (entre ellas, el Muisca), la notable reducción de su númeo de hablantes y la conformación de nuevos procesos lingüísticos; algunos grupos indígenas perdieron su lengua ancestral [..] adoptando el español, y nuevas realidades y necesidades lingüisticas afloraron en el territorio nacional} \cite[157]{Pachon1997}.

\enquote{La Constitución del 86 [...] condensaba una visión unitaria de Colombia, so sólo política sino cultural: Colombia fue definida como un país de tradición hispanica y católica} \cite[158]{Pachon1997}.

In 1890 laws were passed to establish how to govern in order to ensure that \textit{los salvajes} are brought into the fold of civilized life \cite[158]{Pachon1997}. As Pachón highlights, \enquote{en este contexto, el aprendizaje del español constituyó uno de los criterios de reducirse a la vida civilizada} \cite[158]{Pachon1997}. Schooling and Catholic missions carried out the role of teaching Spanish with the goal of eradicating non-Colombian, indigenous, cultures. 

According to Montes, \enquote{el influjo indígena en el español de Colombia es relativamente débil y que no representa una porción notable ni del léxico usual y básico y menos aún de la estructura fónica o gramatical} \cite[72]{Montes1997}.

\enquote{Aunque actualmente el plurilingüismo se considera como una expresión de la riqueza cultural de un país, aquel conllevó durante la colonia y siglo XIX un cierto estigma por el número de problemas de orden político y administrativo que incidían negativa y desfavorablemente contra las propias lenguas} \cite[136]{Atorveza1997}

pg 264 references Isaacs essay
crossref at 
https://books.google.com/books?id=efa8IM-lwmAC&lpg=PR18&ots=PYlFk9v9Uu&dq=estudio%20sobre%20las%20tribus%20ind%C3%ADgena%20del%20magdalena&pg=PR18#v=onepage&q=estudio%20sobre%20las%20tribus%20ind%C3%ADgenas%20del%20magdalena&f=false

online book
http://www.banrepcultural.org/blaavirtual/antropologia/tribus/indice.htm
