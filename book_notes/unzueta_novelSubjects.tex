mechanisms of subject formation within broader nation articulations by analyzing different ways and scenes of reading \cite[75]{Unzueta2002}.

"nineteenth-century novels introduce a radical shift and an opening of the "scenes of reading" in Latin America" \cite[78]{Unzueta...}

"This shift is related to changes in the cultural, publishing, reading and writing practices of that era, and provides for the closer identification between readers and characters, the public and the nation" \cite[78]{Unzueta}

%this is the perfect place to note the importance of the definition of the word nation. Chiaramonte

"El Periquillo sarniento already inaugurated a new type of reading scene in the continent's letters, thanks to three related phenomena"
1. this novel opened itself to a much wider readership
2. its publication and dissemination was closely connected to newspapers
3. it entered into a market of consumer goods, with cultural goods increasingly among them

"the commercial nature of novels, for instance, clearly signalled to their (relatively) widespread readership" \cite[79]{Unzueta2002}.
Unzueta reconciles mass illiteracy with this popularity by noting the historical importance of public readings
and notes "the novels incirporations of these themes and strategies" \cite[79]{Unzueta2002}.
%this could help attribute the likelihood of a novel becoming a performance piece such as Maria s20 

Unzueta reccomends Sarmientos "Nuestro pecado de los folletines" as a description of the beginnings of folletines

"It is very likely that Soledad reached a much wider audience than El Periquillo when each work first appeared"
FN "El Periquillo however, had a much more (re-)productive subsequent history. While Soledad was not re-edited until 1907, Lizardi's work went through nine editions by that time" \cite[79]{Unzueta2002}.

follow up paragraph 1 page 80 and footnote

"Continuing an Enlightenment tradition, most of the novelists of this century, from Lizardi to Matto de Turner, sought to educate the people, to improve their costumbres, and to better their societies. The novelty is that they wrote wider public thanks to the genre's larger circulation, and to the languages and styles they used" \cite[80]{Unzueta2002}.

"reader identification with a text's protagonists, its national contents, and values is crucial to the construction of an imagined community" \cite[82]{Unzueta2002}

The epic poems of the early nineteenth century were "hero-worship" and "homages to national or Latin American independence" \cite[82]{Unzueta2002}

He continues "thanks to their tone and contents, these poems and their heroes acquire a certain monumentality that is easily recognizable but not conducive to personal identification" \cite[82]{Unzueta2002}

This all changed with Periquillo which contained a mixture of writing and orality \cite[82]{Unzueta2002}

"whether reading romances or listening to nationalist discourses and patriotic stories actually changed people's and not just fictional characters' lives" \cite[89]{Unzueta2002}

compare these trends to biography to verify (examples on page 89)
%this is where using the topics and keywords against the corpus will "uncover" proof maybe in what texts we find




