Link to a 1981 ad for Maria in El Tiempo
https://news.google.com/newspapers?id=jTM0AAAAIBAJ&sjid=G2EEAAAAIBAJ&pg=3806%2C1821430

Monumento a Isaacs 1916
https://news.google.com/newspapers?id=fMsdAAAAIBAJ&sjid=KKUEAAAAIBAJ&pg=1503%2C2778149

Capítulo I
Empieza con un recuerdo de cuando alejó de su casa para ir a estudiar de joven
cuenta la despedida de su madre, hermanas y María

Capítulo II (54)
Seis años más tarde regresa a su “nativo valle” (54)   -hipérbaton 
“Mi corazón rebosaba de amor patrio” (54)
En camino a al casa de sus padres describe el paisaje en detalle
describe como sentía cuando vio a la familia 

Capítulo III (56)
Cenan juntos, describe en detalle el aspecto de María 
Lo muestran su cuarto preparado
Hablan de las flores que María ha puesto en su cuarto, que se repondrán cada día

Capitulo IV (58)
Recuerda dormir en su niñez de los cuentos de Pedro
Soñó de María
Ve a María en el jardín descalzo con su hermana
Describe Bogotá a las mujeres en su costurero
Lo preparan un baño de flores

Capítulo V (60)
Visita las haciendas del valle de su padre
Explica que su padre “daba un tratamiento cariñoso a sus esclavos” (61)
habla de los labradores Higinio, Bruno y su esposa Remigia
Describe la boda, la musica de los “negritos” etc. (62)
Destaca el contraste del dialecto “mano, que así llaman los campesinos cada pieza de baile” (63)
Su padre dice que va a enviarlo a Europa para terminar los estudios

Capítulo VI (63)
Regresa de las haciendas y nota un cambio en María
Hay un aparte que exalta su amor

Capítulo VII (65)
La historia de la llegada de María y la muerte de su madre a las tres años
Dscribe su conversión a la cristianidad y la muerte de su padre a las nueve años

Capítulo VIII (68)
Cena con sus padres pero María no viene
Duda su amor, “consideréme indigno de poseer tanta belleza, tanta innocencia” (68)

Capítulo IX (69)
Más descripción geográfica, palabras distintas
Viaja a la casa del viejo José
Habla de “la faz bíblica” de José en referencia a la leyenda de la orígen hebreo de los antioqueños (71 fn)
Describe que su mujer “conservaba en el vestir algo de la manera antioqueña” (71)
Almmuerza con José y regalales recuerdos a todos
“reino” = cundinamarca (72)

Capítulo X (72)
Regresa a casa y ve a María está allí. 
Ella no devuelve la mirada de él.
Ha traído flores pero no puede darselas a ella.



Capítulo XI (73)
¿Qué es el fenómeno del P*** en apellidos?
María caricia Juan, el hermano de tres años
Sienten todos en el comedor y María tiene una de las azucenas en su caballera
El padre le pregunta ¿de dónde es?
Más tarde Efraín le confesa a María sus intenciones
Ella promete recogerle las flores más lindas

Capítulo XII (75)
Efraín empieza enseñar a las mujeres 
Describe los sentimientos de estar cerca a María en las lecciones
Revela que habla de un futuro en el que no ha oído su voz hace muchos años

Capítulo XIII (78)
María se dedica a la lectura religiosa
Efraín describe cómo leyeron juntos un poema de amor y se unieron

Capítulo XIV (79)
María sufrió un ataque nervioso
Juan llora sobre ella y de repente se despierta y dice su nombre
se queda inmóvil en el lecho pero habla y se despide a Juan “hasta mañana” (81)

Capítulo XV (81)
Mira una tormenta en el patio
Nota McGrady de la frase “naturaleza sollozante” que “el cuadro anterior (página 81) constituye una ilustación excelente de cómo la naturaleza muchas veces refleja el estado anímico del protagonista en la ficción romantica” (82).
El padre informa a Efraín que ha peorado María
Él sale para el doctor Mayn
Larga descripción de la travesía, el cruce del río

Capítulo XVI (84)
Los padres hablan con Efraín de los obstaculos de su amor
Dice que María va a morir como su madre
Que sus síntomas son a causa de su amor
Revela que Señor M*** quiere que su hijo se case con María
Efraín cancela su cacería de osos y lamenta su dilema mientras mira la tormenta

Capítulo XVII (91)
Se deprime Efraín pensando en el dilema y ignora a María
Su madre lo visita para regañarle por su comportamiento 
Le dice a Efraín que doctor Mayn cambió su pronóstico
Efraín le asegura que va a remediar la situación dentro de dos días

Capítulo XVIII (94)
Emma visita a Efraín y se peine 
Hablan de la hermana de Emigdio
El padre le da a Efraín un reloj de bolsa
Efraín habla con María en el jardín

Capítulo XIX (97) -especialmente rico de vocabulario
Empieza con una descripción llena de vocabulario distinto (97)
“se aparaguan”, “neologísmo de Isaacs” (98)
Cuenta del apartamento en Bogotá con Carlos
	Los visita Emigdio, introduce Micaela
Efraín visita la hacienda de la familia de Emigdio
Varias menciones despectivas de esclavos/negros
“el sudor peculiar de la raza” (98)
“son tan brutos estos, no sirve ya sino para cuidar a los caballos” (102)
Después de almuerzo van a ver don Ignacio
Se bañan en el rio donde Emigdio confiesa que ama a un ñapango(mestiza) Zoila
Efraín regre9sa a casa 

Capítulo XX (107)
Madre y María le gritan a Efraín 
Él va para limpiar su escopete 
María le trae té y le ayuda recoger unas partes caídas
Ellos hacen un pacto sobre su amor

Capítulo XXI (110)
Efraín va a la cas de José para la caza del tigre
Cenan con las mujeres y luego van al bosque
Encuentran el tigre pero Braulio resulta en peligro de ser herido cuando Efraín dispara y mata el león

hay un aparte en parenteses en la página 118 por la primera vez
también hay vocabulario muy descriptivo coloquial 

XXII (122)
Regresa a casa Efraín,
Encuentran a Juan Angel en camino asusutado de las noticias que oyó de Lucas
Entra en secreto a su cuarto, habla con su madre
juan Angel entrega la cabeza del leon al salón 
Hablan su padre, don Jeronimo, Carlos  las mujeres de la cacería
Carlos y Efraín van a su cuarto y hablan
Carlos "fiscaliza" la biblioteca de Efraín

parece mucho comentario sobre Carlos, critica el lenguaje (130)

XXIII (131)
Todos comen juntos
Retiran al salón para cantar
Cantan unos versos de Efraín pero él miente a Carlos de sus orígenes
Critica más el narrador a Carlos

XXIV (136)
Efraín duerme en el cuarto de los niños 

XXV (138)
Despierta, jugando con los niños
Llega Braulio con los perros para la caza
Carlos bromea que los perros son sarnosos y no buenos para la caza
Su madre propone que hablen con María de la situación
La madre le explica a María que Carlos quiere casaese con ella, etc.
María dice que no y hacen un plan para rechazarlo

XXVI (145)
Carlos, Braulio y Efraín van a cazar un venado cerca de la casa
Efraín deja que Carlos dispara al animal pero Carlos falla
el venado llega hasta el salon de la casa donde se derrumba
antes de irse Braulio admite a Efraín que no cargó la escopeta de Carlos para vengarse del insulto previo

XXVII (149)
Carlos y Efraín van a su cuarto
Carlos lamenta la caza
don J viene para traer a Carlos a María
Efraín habla con María y luego se baña bajo los naranjos
Descripción de Estefana, nota McGrady que es similar a la de Juan (153)
don J comenta en cómo vive Efraín 

XXVIII (154)
Todos van a caminar en la vega donde Carlos pregunta a María
María dice que no
El padre de Efraín habla con María
Carlos y Efraín regresan a su cuarto y hablan con franqueza
Efraín revela toda la verdad a Carlos
Efraín y María hablan en el salón "nunca se había mostrado tan expansiva conmigo" (160)

XXIX
María y Efraín hablan en el salón minetras juegan damas
María quiere saber si ha dicho a Carlos
Efraín explica todo
al final el narrador alude al futuro "hasta que vinieron los días en que se mezclaban tantas veces nuestras lágrimas " (163)

XXX (163)
Efraín y su padre trabajan en su oficina
María viene para cortar el pelo del padre
Se cae una rosa de su cabello y la repone su padre 

XXXI (166)
Al principio del capítulo Efraín lamenta la visión que recuerda de su cuarto "no era eso que lo veían mis ojos; era lo que no veré más; lo que mi espíritu quebrantado por tristes realidades no busca" (166)

Visitan José y Tránsito para hablar del matrimonio entre ella y Braulio
Hace referencia a las divisiones raciales "en la provincia solamente los blancos andan a caballo" "¿Quién te ha dicho que no eres blanca?" (168)
Tambien comenta otra vez en la lengua de Juan "me contestó en aquella lengua que pocos podían entenderle" (170)
Después de la visita hablan Efraín y María de un bucle de su pelo que le ha pedido Efraín
Juan queja y quiere que Efraín se haga dormir
María juega con Juan y él se pone alegre
Ella da a Efraín el bucle y pide uno de él 

XXXII (173)
El padre juega con María minetras ella empaquea por el viaje
María corta un pedazo de pelo de Efraín para su guarda-pelo

XXXIII (175)
Han pasado los siete días del viaje
llega una carta con malas noticias
el Padre lamenta su falta de desconfianza con los hombres
exige que Efraín se oculte lo que ha pasado
regresan a casa

XXXIV (179)
Llegan a la casa y encuentran María y Emma jugando encima de una roca
María no puede bajar y Efraín se queda para ayudarla
Después hablan del viaje y como se extrañaron
María comparte la historia de una ave negra que Eraín toma por agüero

XXXV (183)
Van todos a la boda de Braulio y Tránsito 
María monta a caballo y bromea con Efraín
Hablan ellos del deseo secreto que tiene María de que Efraín tenga que quedarse debido al fracaso de su padre
Eligen que él va a ofrecer su ayuda a su padre

Comentario en como se veía el índole de la mujer "Yo soy una muchacha capaz, como cualquiera otra, de no ver las cosas serias como deben verse" (189)
good example of the indoctrination of a culture of religious monogamy as noted by Sommer "en aquella mirada había amor, humildad e inocencia; era la promesa única que podía hacer al hombre que amaba después de la que acababa de pronunciar ante Dios" (190).

XXXVI (192)
Regresan a la casa para encontrar su padre enfermo
Llaman al doctor y Efraín explica en privado que ha ocurrido en el negocio
Su madre oye algo que murmuró el padre a Efraín pero él lo niega como si fuera halucinación de la fiebre
Le dan una poción narcótica 
termina diciendo "el médico estaba visiblemente preocupado" (198)



Personajes:
el Yo: Efraín
su padre:
su madre:
las hermanas: Emma, Eloísa?
El hermano: Juan
María (Ester antes de su bautismo)
Lorenzo María Lleras
el esclavo Pedro
Higinio, el mayordomo de las haciendas
Bruno
Remigia, esposa de Bruno
Dolores y Anselmo, padrinos en la boda de Bruno
Salomón, padre de María
Sara, la esposa muerta de Salomón
Mayo, el perro de Efraín
el viejo José
Lucía y Tránsito hijas de José
Luisa, esposa de José
el negrito Juan Ángel
doctor Mayn
Braulio, sobrino de José
Don Ignacio, padre de Emigdio
Doña Andrea, madre de  “”
Emigdio
Señor M***, padre de Carlos
Carlos, vivía con Efraín en Bogotá
Micaelina, hija del dueño del apartamento de Efraín en Bogotá
el mulatto Tiburcio
Lucas, hijo del vecino de José
Marta, la  cocinera en la casa de José
Palomo, perro favorito de Marta
Juan Ángel, el "negrito" de Efraín
Feliciana, madre de Juan Angel
Estefana, negra de doce años 
