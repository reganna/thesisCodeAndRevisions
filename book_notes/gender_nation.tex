Eva pualino bueno, gender and nation

"I was expected at some point of my education to read María, by Jorge Isaacs. At the
time, the reading of this text, so representative of Romanticism in the Nineteenth
century, was a ritual of passage for anyone completing a K–12 degree
program in the country. The text had been embraced by other Latin American
nations and was repeatedly celebrated as a high point of Colombia’s literature
at a time yet unaffected by the clout of Gabriel García Márquez and magic
realism" \cite[14]{Bueno2012}.

"i found the protracted descriptions of Colombian landscapes wearisome and demanding of my patience" \cite[14]{Bueno2012}.

"  began identifying
some of its narrative as troubling and perilously uneven in its management of
features of national identity" \cite[15]{Bueno2012}.

"María always felt like a male prescription of femininity" \cite[15]{Bueno2012}.

